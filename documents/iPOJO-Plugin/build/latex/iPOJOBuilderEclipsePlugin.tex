% Generated by Sphinx.
\def\sphinxdocclass{report}
\documentclass[letterpaper,10pt,french]{sphinxmanual}
\usepackage[utf8]{inputenc}
\DeclareUnicodeCharacter{00A0}{\nobreakspace}
\usepackage[T1]{fontenc}
\usepackage{babel}
\usepackage{times}
\usepackage[Sonny]{fncychap}
\usepackage{longtable}
\usepackage{sphinx}


\title{iPOJO Builder Eclipse Plugin Documentation}
\date{03 February 2011}
\release{alpha}
\author{Thomas Calmant, Olivier Gattaz}
\newcommand{\sphinxlogo}{}
\renewcommand{\releasename}{Version}
\makeindex

\makeatletter
\def\PYG@reset{\let\PYG@it=\relax \let\PYG@bf=\relax%
    \let\PYG@ul=\relax \let\PYG@tc=\relax%
    \let\PYG@bc=\relax \let\PYG@ff=\relax}
\def\PYG@tok#1{\csname PYG@tok@#1\endcsname}
\def\PYG@toks#1+{\ifx\relax#1\empty\else%
    \PYG@tok{#1}\expandafter\PYG@toks\fi}
\def\PYG@do#1{\PYG@bc{\PYG@tc{\PYG@ul{%
    \PYG@it{\PYG@bf{\PYG@ff{#1}}}}}}}
\def\PYG#1#2{\PYG@reset\PYG@toks#1+\relax+\PYG@do{#2}}

\def\PYG@tok@gd{\def\PYG@tc##1{\textcolor[rgb]{0.63,0.00,0.00}{##1}}}
\def\PYG@tok@gu{\let\PYG@bf=\textbf\def\PYG@tc##1{\textcolor[rgb]{0.50,0.00,0.50}{##1}}}
\def\PYG@tok@gt{\def\PYG@tc##1{\textcolor[rgb]{0.00,0.25,0.82}{##1}}}
\def\PYG@tok@gs{\let\PYG@bf=\textbf}
\def\PYG@tok@gr{\def\PYG@tc##1{\textcolor[rgb]{1.00,0.00,0.00}{##1}}}
\def\PYG@tok@cm{\let\PYG@it=\textit\def\PYG@tc##1{\textcolor[rgb]{0.25,0.50,0.56}{##1}}}
\def\PYG@tok@vg{\def\PYG@tc##1{\textcolor[rgb]{0.73,0.38,0.84}{##1}}}
\def\PYG@tok@m{\def\PYG@tc##1{\textcolor[rgb]{0.13,0.50,0.31}{##1}}}
\def\PYG@tok@mh{\def\PYG@tc##1{\textcolor[rgb]{0.13,0.50,0.31}{##1}}}
\def\PYG@tok@cs{\def\PYG@tc##1{\textcolor[rgb]{0.25,0.50,0.56}{##1}}\def\PYG@bc##1{\colorbox[rgb]{1.00,0.94,0.94}{##1}}}
\def\PYG@tok@ge{\let\PYG@it=\textit}
\def\PYG@tok@vc{\def\PYG@tc##1{\textcolor[rgb]{0.73,0.38,0.84}{##1}}}
\def\PYG@tok@il{\def\PYG@tc##1{\textcolor[rgb]{0.13,0.50,0.31}{##1}}}
\def\PYG@tok@go{\def\PYG@tc##1{\textcolor[rgb]{0.19,0.19,0.19}{##1}}}
\def\PYG@tok@cp{\def\PYG@tc##1{\textcolor[rgb]{0.00,0.44,0.13}{##1}}}
\def\PYG@tok@gi{\def\PYG@tc##1{\textcolor[rgb]{0.00,0.63,0.00}{##1}}}
\def\PYG@tok@gh{\let\PYG@bf=\textbf\def\PYG@tc##1{\textcolor[rgb]{0.00,0.00,0.50}{##1}}}
\def\PYG@tok@ni{\let\PYG@bf=\textbf\def\PYG@tc##1{\textcolor[rgb]{0.84,0.33,0.22}{##1}}}
\def\PYG@tok@nl{\let\PYG@bf=\textbf\def\PYG@tc##1{\textcolor[rgb]{0.00,0.13,0.44}{##1}}}
\def\PYG@tok@nn{\let\PYG@bf=\textbf\def\PYG@tc##1{\textcolor[rgb]{0.05,0.52,0.71}{##1}}}
\def\PYG@tok@no{\def\PYG@tc##1{\textcolor[rgb]{0.38,0.68,0.84}{##1}}}
\def\PYG@tok@na{\def\PYG@tc##1{\textcolor[rgb]{0.25,0.44,0.63}{##1}}}
\def\PYG@tok@nb{\def\PYG@tc##1{\textcolor[rgb]{0.00,0.44,0.13}{##1}}}
\def\PYG@tok@nc{\let\PYG@bf=\textbf\def\PYG@tc##1{\textcolor[rgb]{0.05,0.52,0.71}{##1}}}
\def\PYG@tok@nd{\let\PYG@bf=\textbf\def\PYG@tc##1{\textcolor[rgb]{0.33,0.33,0.33}{##1}}}
\def\PYG@tok@ne{\def\PYG@tc##1{\textcolor[rgb]{0.00,0.44,0.13}{##1}}}
\def\PYG@tok@nf{\def\PYG@tc##1{\textcolor[rgb]{0.02,0.16,0.49}{##1}}}
\def\PYG@tok@si{\let\PYG@it=\textit\def\PYG@tc##1{\textcolor[rgb]{0.44,0.63,0.82}{##1}}}
\def\PYG@tok@s2{\def\PYG@tc##1{\textcolor[rgb]{0.25,0.44,0.63}{##1}}}
\def\PYG@tok@vi{\def\PYG@tc##1{\textcolor[rgb]{0.73,0.38,0.84}{##1}}}
\def\PYG@tok@nt{\let\PYG@bf=\textbf\def\PYG@tc##1{\textcolor[rgb]{0.02,0.16,0.45}{##1}}}
\def\PYG@tok@nv{\def\PYG@tc##1{\textcolor[rgb]{0.73,0.38,0.84}{##1}}}
\def\PYG@tok@s1{\def\PYG@tc##1{\textcolor[rgb]{0.25,0.44,0.63}{##1}}}
\def\PYG@tok@gp{\let\PYG@bf=\textbf\def\PYG@tc##1{\textcolor[rgb]{0.78,0.36,0.04}{##1}}}
\def\PYG@tok@sh{\def\PYG@tc##1{\textcolor[rgb]{0.25,0.44,0.63}{##1}}}
\def\PYG@tok@ow{\let\PYG@bf=\textbf\def\PYG@tc##1{\textcolor[rgb]{0.00,0.44,0.13}{##1}}}
\def\PYG@tok@sx{\def\PYG@tc##1{\textcolor[rgb]{0.78,0.36,0.04}{##1}}}
\def\PYG@tok@bp{\def\PYG@tc##1{\textcolor[rgb]{0.00,0.44,0.13}{##1}}}
\def\PYG@tok@c1{\let\PYG@it=\textit\def\PYG@tc##1{\textcolor[rgb]{0.25,0.50,0.56}{##1}}}
\def\PYG@tok@kc{\let\PYG@bf=\textbf\def\PYG@tc##1{\textcolor[rgb]{0.00,0.44,0.13}{##1}}}
\def\PYG@tok@c{\let\PYG@it=\textit\def\PYG@tc##1{\textcolor[rgb]{0.25,0.50,0.56}{##1}}}
\def\PYG@tok@mf{\def\PYG@tc##1{\textcolor[rgb]{0.13,0.50,0.31}{##1}}}
\def\PYG@tok@err{\def\PYG@bc##1{\fcolorbox[rgb]{1.00,0.00,0.00}{1,1,1}{##1}}}
\def\PYG@tok@kd{\let\PYG@bf=\textbf\def\PYG@tc##1{\textcolor[rgb]{0.00,0.44,0.13}{##1}}}
\def\PYG@tok@ss{\def\PYG@tc##1{\textcolor[rgb]{0.32,0.47,0.09}{##1}}}
\def\PYG@tok@sr{\def\PYG@tc##1{\textcolor[rgb]{0.14,0.33,0.53}{##1}}}
\def\PYG@tok@mo{\def\PYG@tc##1{\textcolor[rgb]{0.13,0.50,0.31}{##1}}}
\def\PYG@tok@mi{\def\PYG@tc##1{\textcolor[rgb]{0.13,0.50,0.31}{##1}}}
\def\PYG@tok@kn{\let\PYG@bf=\textbf\def\PYG@tc##1{\textcolor[rgb]{0.00,0.44,0.13}{##1}}}
\def\PYG@tok@o{\def\PYG@tc##1{\textcolor[rgb]{0.40,0.40,0.40}{##1}}}
\def\PYG@tok@kr{\let\PYG@bf=\textbf\def\PYG@tc##1{\textcolor[rgb]{0.00,0.44,0.13}{##1}}}
\def\PYG@tok@s{\def\PYG@tc##1{\textcolor[rgb]{0.25,0.44,0.63}{##1}}}
\def\PYG@tok@kp{\def\PYG@tc##1{\textcolor[rgb]{0.00,0.44,0.13}{##1}}}
\def\PYG@tok@w{\def\PYG@tc##1{\textcolor[rgb]{0.73,0.73,0.73}{##1}}}
\def\PYG@tok@kt{\def\PYG@tc##1{\textcolor[rgb]{0.56,0.13,0.00}{##1}}}
\def\PYG@tok@sc{\def\PYG@tc##1{\textcolor[rgb]{0.25,0.44,0.63}{##1}}}
\def\PYG@tok@sb{\def\PYG@tc##1{\textcolor[rgb]{0.25,0.44,0.63}{##1}}}
\def\PYG@tok@k{\let\PYG@bf=\textbf\def\PYG@tc##1{\textcolor[rgb]{0.00,0.44,0.13}{##1}}}
\def\PYG@tok@se{\let\PYG@bf=\textbf\def\PYG@tc##1{\textcolor[rgb]{0.25,0.44,0.63}{##1}}}
\def\PYG@tok@sd{\let\PYG@it=\textit\def\PYG@tc##1{\textcolor[rgb]{0.25,0.44,0.63}{##1}}}

\def\PYGZbs{\char`\\}
\def\PYGZus{\char`\_}
\def\PYGZob{\char`\{}
\def\PYGZcb{\char`\}}
\def\PYGZca{\char`\^}
\def\PYGZsh{\char`\#}
\def\PYGZpc{\char`\%}
\def\PYGZdl{\char`\$}
\def\PYGZti{\char`\~}
% for compatibility with earlier versions
\def\PYGZat{@}
\def\PYGZlb{[}
\def\PYGZrb{]}
\makeatother

\begin{document}

\maketitle
\tableofcontents
\phantomsection\label{index::doc}


Contenu :


\chapter{Références de développement Eclipse}
\label{eclipse/index:documentation-du-plugin-eclipse-ipojo-builder}\label{eclipse/index::doc}\label{eclipse/index:references-de-developpement-eclipse}

\section{Références de développement}
\label{eclipse/index:references-de-developpement}

\subsection{Intéractions avec le workspace}
\label{eclipse/index:interactions-avec-le-workspace}
Modifications de ressources :
\begin{itemize}
\item {} 
\href{http://help.eclipse.org/helios/index.jsp?topic=/org.eclipse.platform.doc.isv/guide/resAdv\_events.htm}{Suivi des modifications de ressources}

\item {} 
\href{http://help.eclipse.org/helios/index.jsp?topic=/org.eclipse.platform.doc.isv/guide/resAdv\_batching.htm}{Traitement des modifications de ressources}

\item {} 
\href{http://help.eclipse.org/helios/index.jsp?topic=/org.eclipse.platform.doc.isv/guide/resAdv\_hooks.htm}{Hook sur les modifications de ressources}

\item {} 
\href{http://help.eclipse.org/helios/index.jsp?topic=/org.eclipse.platform.doc.isv/guide/resAdv\_derived.htm}{Ressource ``dérivées''}

\end{itemize}

Gestion de projet :
\begin{itemize}
\item {} 
\href{http://help.eclipse.org/helios/index.jsp?topic=/org.eclipse.platform.doc.isv/guide/resAdv\_natures.htm}{Nature de projet}

\end{itemize}

Builder :
\begin{itemize}
\item {} 
\href{http://help.eclipse.org/helios/index.jsp?topic=/org.eclipse.platform.doc.isv/guide/resAdv\_builders.htm}{Builder incrémental}

\end{itemize}


\subsection{Intéractions avec JDT}
\label{eclipse/index:interactions-avec-jdt}
Quelques références pouvant être utiles :
\begin{itemize}
\item {} 
\href{http://help.eclipse.org/helios/index.jsp?topic=/org.eclipse.jdt.doc.isv/guide/jdt\_api\_run.htm}{Exécution d'un programme Java}

\end{itemize}


\subsection{Builder Eclipse}
\label{eclipse/index:builder-eclipse}

\subsubsection{Nature de projet}
\label{eclipse/index:id1}
Un builder est associé à un projet en passant par la notion de \emph{nature}.
Cette notion permet également d'associer un projet à des plugins.

L'association d'un builder à un projet se fait à l'aide de la méthode
\emph{configure()}.

\begin{tabulary}{\linewidth}{|L|L|}
\hline

\textbf{Point d'extension}
 & 
org.eclipse.core.resources.natures
\\

\textbf{Interface}
 & 
IProjectNature
\\

\textbf{Références}
 & 
Page 123-136 du diaporama de M. Baron ;
\\
\hline
\end{tabulary}



\subsubsection{Builder}
\label{eclipse/index:builder}
Le compilateur en lui-même.
Il existe un modèle fourni par Eclipse sur lequel on peut s'inspirer.

\begin{tabulary}{\linewidth}{|L|L|}
\hline

\textbf{Point d'extension}
 & 
org.eclipse.core.resources.builders
\\

\textbf{Interface}
 & 
IncrementalProjectBuilder (classe)
\\

\textbf{Références}
 & 
Page 108-122 du diaporama de M. Baron ;
\\
\hline
\end{tabulary}



\subsection{Éditeur de fichier metadata.xml}
\label{eclipse/index:editeur-de-fichier-metadata-xml}
Voir les références utilisées pour le plugin \textbf{ReST Editor}


\subsection{Liens utiles pour iPOJO}
\label{eclipse/index:liens-utiles-pour-ipojo}\begin{itemize}
\item {} 
\href{http://felix.apache.org/site/dive-into-the-ipojo-manipulation-depths.html}{Principe de la manipulation}

\end{itemize}


\chapter{Document développeur}
\label{developer/index::doc}\label{developer/index:document-developpeur}

\section{Description du plugin}
\label{developer/index:description-du-plugin}
Ce projet est un plugin pour Eclipse Helios permettant d'utiliser facilement
iPOJO dans cet environnement de développement.
Il fournit une nouvelle nature de projet et un ``builder'' associé à cette nature.

Le but n'est pas (actuellement) d'assister l'utilisateur dans la création des
fichiers de description iPOJO (metadata.xml, annotations, ...) mais de lui
permettre d'utiliser des bundles iPOJO dans ses configurations d'exécution sans
avoir à passer par Maven ou par un fichier JAR à placer dans une
``Target Platform'' de test.


\section{Outils existants}
\label{developer/index:outils-existants}
Nous avons trouvé deux outils concernant l'intégration d'iPOJO dans Eclipse :
\begin{itemize}
\item {} 
Le plugin Eclipse fourni par Apache Felix
Site Web: \href{http://felix.apache.org/site/ipojo-eclipse-plug-in.html}{http://felix.apache.org/site/ipojo-eclipse-plug-in.html}
Il permet d'exporter le projet sous forme d'un JAR traité par iPOJO.
Géré par Clément ESCOFFIER, il ne semble pas avoir évolué depuis 2008.

\item {} 
Le builder iPOJO du projet CADSE du laboratoire ADELE (Grenoble)
Site Web : \href{http://code.google.com/a/eclipselabs.org/p/cadse/}{http://code.google.com/a/eclipselabs.org/p/cadse/}
Découvert sur le tard, il s'agit d'un projet équivalent au notre.
Géré par Stéphane CHOMAT, la dernière mise à jour date de Septembre 2010.

\end{itemize}


\section{Principe de fonctionnement}
\label{developer/index:principe-de-fonctionnement}
Le principe du plugin est d'ajouter la nature ``iPOJO'' à un projet existant,
ajoutant notre builder à sa liste.
Ce plugin ajoute également une entrée ``Update Manifest'' dans le menu contextuel
des fichiers Manifest.mf, permettant de faire une ``Pojoization'' manuelle.

Lorsque le builder est appelé, ou qu'une mise à jour manuelle est demandée, le
plugin demande une compilation complète du projet au plugin JDT, puis utilise
l'outil iPOJO Manipulator pour effectuer le traitement des fichiers .class
générés et du fichier Manifest.
Il ne s'agit pour le moment que d'une recherche des fichiers dans le projet et
le format Eclipse pour les transmettre au Manipulator, utilisant les interfaces
Java standards.


\section{Fonctionnalités du plugin}
\label{developer/index:fonctionnalites-du-plugin}

\subsection{Fonctionnalités attendues}
\label{developer/index:fonctionnalites-attendues}

\subsubsection{Pojoization automatique des fichiers .class}
\label{developer/index:pojoization-automatique-des-fichiers-class}\begin{itemize}
\item {} 
Les fichiers class doivent être \emph{pojoizés} avant l'export de fichier JAR
ou l'exécution d'une \emph{Run Configuration}.

\item {} 
Le meilleur moyen d'assurer cette fonctionnalité est d'effectuer le
traitement sur les fichiers class fraîchement compilés par JDT.

\end{itemize}


\paragraph{Builder Eclipse pour la Pojoization}
\label{developer/index:builder-eclipse-pour-la-pojoization}
Le plugin doit fournir un ``builder'' s'insérant dans la chaîne de compilation
d'un projet Java utilisant ou non Maven.

Ce builder doit modifier les fichier class et le fichier Manifest.mf du projet
dès qu'un de ces fichiers a été modifié par un autre compilateur ou par
l'utilisateur.

Le builder peut être soit un plugin Eclipse pur, soit un plugin JDT.


\paragraph{Compatibilité avec le plugin Maven}
\label{developer/index:compatibilite-avec-le-plugin-maven}
Le plugin iPOJO doit être capable d'interagir avec le plugin Maven pour Eclipse.
Ces deux plugins effectuent des opérations sur les fichiers class dès qu'une
modification a eu lieu dans un fichier du projet, ce qui pourrait entraîner une
boucle sans fin.


\subsection{Fonctionnalités optionnelles}
\label{developer/index:fonctionnalites-optionnelles}

\subsubsection{Éditeur de fichier metadata.xml}
\label{developer/index:editeur-de-fichier-metadata-xml}\begin{itemize}
\item {} 
Au moins fournir un template pour l'éditeur XML, avec les XML schemas
renseignés dans le prototype.

\item {} 
Ajouter la completion des noms java, des noms de composant et de propriétés
connus

\end{itemize}


\chapter{Documentation utilisateur}
\label{user/index:documentation-utilisateur}\label{user/index::doc}

\section{Description du plugin}
\label{user/index:description-du-plugin}
Ce projet est un plugin pour Eclipse Helios permettant d'utiliser facilement
iPOJO dans cet environnement de développement.
Il fournit une nouvelle nature de projet et un ``builder'' associé à cette nature.

Le but n'est pas (actuellement) d'assister l'utilisateur dans la création des
fichiers de description iPOJO (metadata.xml, annotations, ...) mais de lui
permettre d'utiliser des bundles iPOJO dans ses configurations d'exécution sans
avoir à passer par Maven ou par un fichier JAR à placer dans une
``Target Platform'' de test.


\section{Outils existants}
\label{user/index:outils-existants}
Nous avons trouvé deux outils concernant l'intégration d'iPOJO dans Eclipse :
\begin{itemize}
\item {} 
Le plugin Eclipse fourni par Apache Felix
Site Web: \href{http://felix.apache.org/site/ipojo-eclipse-plug-in.html}{http://felix.apache.org/site/ipojo-eclipse-plug-in.html}
Il permet d'exporter le projet sous forme d'un JAR traité par iPOJO.
Géré par Clément ESCOFFIER, il ne semble pas avoir évolué depuis 2008.

\item {} 
Le builder iPOJO du projet CADSE du laboratoire ADELE (Grenoble)
Site Web : \href{http://code.google.com/a/eclipselabs.org/p/cadse/}{http://code.google.com/a/eclipselabs.org/p/cadse/}
Découvert sur le tard, il s'agit d'un projet équivalent au notre.
Géré par Stéphane CHOMAT, la dernière mise à jour date de Septembre 2010.

\end{itemize}


\section{Principe de fonctionnement}
\label{user/index:principe-de-fonctionnement}
Le principe du plugin est d'ajouter la nature ``iPOJO'' à un projet existant,
ajoutant notre builder à sa liste.
Ce plugin ajoute également une entrée ``Update Manifest'' dans le menu contextuel
des fichiers Manifest.mf, permettant de faire une ``Pojoization'' manuelle.

Lorsque le builder est appelé, ou qu'une mise à jour manuelle est demandée, le
plugin demande une compilation complète du projet au plugin JDT, puis utilise
l'outil iPOJO Manipulator pour effectuer le traitement des fichiers .class
générés et du fichier Manifest.
Il ne s'agit pour le moment que d'une recherche des fichiers dans le projet et
le format Eclipse pour les transmettre au Manipulator, utilisant les interfaces
Java standards.


\chapter{Index et tables}
\label{index:index-et-tables}\begin{itemize}
\item {} 
\code{content/glossaire}

\item {} 
\emph{genindex}

\item {} 
\emph{search}

\end{itemize}



\renewcommand{\indexname}{Index}
\printindex
\end{document}
